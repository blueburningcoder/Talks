\documentclass{beamer}
\usepackage[T1]{fontenc}
\usepackage[utf8]{inputenc}
\usepackage[ngerman]{babel}
\usepackage{amsmath}
\usepackage{amssymb}
\usepackage{graphicx}
\usepackage{listings}
\usepackage{color}
\setbeamercovered{transparent}
\setbeamertemplate{footline}[frame number]

%meta
\title{\textbf{Lineare Logik} \\ Of course linear! Why not intuitionistic?}
\author{\textbf{uwap}}
\date{\textbf{\today}}
\begin{document}

\begin{frame}
  \titlepage%
\end{frame}

\begin{frame}
  \frametitle{Inhaltsverzeichnis}
  \tableofcontents
\end{frame}

\section{Lollis und andere Süßigkeiten}
\begin{frame}
  \textbf{Aussage A:} Emily hat genau 5 cent.\\ \pause%
  \textbf{Aussage B:} Emily hat ein Bonbon.\\ \pause%
  Ein Bonbon kostet 5 cent.\\ \pause%

  \begin{equation}
    \frac{A \implies B}
         {\Gamma, A \vdash \Gamma, B}
  \end{equation}
  
  \pause%
  
  \begin{equation}
    \frac{\Gamma \vdash B}
         {\Gamma, A \vdash B}
  \end{equation}
  
  \pause%
  
  \begin{equation}
    \frac{A \implies B}
         {\Gamma, A \vdash \Gamma, A, B}
  \end{equation}
\end{frame}

\begin{frame}
  Contraction
  \begin{equation*}
    \frac{\Gamma, A, A \vdash B}
         {\Gamma, A \vdash B}
  \end{equation*}

  Weakening
  \begin{equation*}
    \frac{\Gamma \vdash B}
         {\Gamma, A \vdash B}
  \end{equation*}
\end{frame}

\end{document}
